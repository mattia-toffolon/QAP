In questo lavoro è stata studiata una particolare classe di problemi di assegnamento quadratico, quella delle istanze Tai*c.
Nello specifico è stato analizzato il tempo medio di risoluzione al variare della dimensione dell'istanza considerata 
e di altri valori utili alla generazione delle istanze stesse.

È stato possibile determinare un andamento generale dei tempi in funzione della dimensione dell'istanza, osservando però come 
questo differisca parzialmente in caso di variazione dei restanti parametri presi in esame.

Tuttavia, le sperimentazioni sono state limitate in primis dal time limit impostato e in secondo luogo dalla potenza computazionale 
dell'hardware utilizzato. Dunque, per ottenere nuovi dati che possano confermare o meno i risultati ottenuti, è possibile aumentare il time limit ed in aggiunta o in alternativa 
utilizzare un calcolatore più potente.

Per ottenere invece prestazioni migliori di quelle riscontrate a parità di dimensione d'istanza e hardware utilizzato, sarebbe 
necassario abbandonare la classe Tai*c ed individuarne una di nuova che porti all'utilizzo di un modello più efficiente.