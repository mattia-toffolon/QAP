A seguito di una prima parte teorica introduttiva, si procede in questo capitolo con la presentazione 
dell'impostazione pratica che si è voluto dare alle sperimentazioni condotte. La struttura secondo la quale verranno esposte 
le informazioni nei seguenti paragrafi ricalca la partizione logica alla base del codice sviluppato, pragmaticamente diviso 
in quattro moduli tra loro indipendenti:
\begin{itemize}
    \item Generazione dei parametri
    \item Generazione delle istanze di problemi MIP
    \item Risoluzione delle istanze di problemi
    \item Estrazione ed analisi dei risultati
\end{itemize}




\section{Generazione dei parametri}
Il primo problema presentatosi è stato quello relativo all'individuamento dei parametri di flusso e distanza, in quanto 
fondamentali per la creazione di istanze del problema e conseguentemente anche per la loro risoluzione.

La prima possibilità valutata è stata la generazione di valori casuali per comporre le matrici dei parametri $A$ e $B$. 
Questa però è stato scartata fin da subito poichè utilizzare valori di tale tipologia non permette di fornire una versione 
pseudo-realistica di istanze del problema ed inoltre, non garantisce alcun tipo di uniformità nella generazione delle istanze, 
il che non ci permette di effettuare successivamente uno studio approfondito sulla complessità di risoluzione di queste.

La soluzione a questo problema è stata individuata in un articolo del professor Éric D. Taillard del 1995
\cite{TAILLARD}. Come verrà illustrato più nello specifico nella prossima sezione, il metodo utilizzato è detto Densità di
grigio, in inglese (\textit{Density of grey}). Esso, oltre a compensare i difetti del metodo di istanziamento casuale, permette
di automatizzare la creazione delle matrici dei parametri realizzando un algoritmo che richiede ai fini della generazione 
esclusivamente due valori: dimensione dell'istanza e densità utilizzata.

Per realizzare tale processo è stato fatto uso del linguaggio di programmazione \textit{Python} \cite{python}, come anticipato
nell'introduzione, e come supporto \textit{NumPy} \cite{NumPy}, una libreria open source che aggiunge supporto a grandi matrici 
e array multidimensionali insieme a una vasta collezione di funzioni matematiche di alto livello per poter operare efficientemente 
su queste strutture dati.

\subsection{Istanze Tai*c}
prova




\section{Generazione delle istanze di problemi MIP}
prova

\subsection{Modellazione algebrica}
prova

\subsection{Linearizzazione del modello}
prova

\subsection{Semplificazione del modello}
prova

\subsection{Modellazione nel software}
prova




\section{Risoluzione delle istanze di problemi}
prova




\section{Estrazione ed analisi dei risultati}
prova