Conclusa la presentazione delle scelte adottate in fase di impostazione delle sperimentazioni, 
si procede in questo capitolo con l'esposizione dei risultati ottenuti. 

Per effettuare la risoluzione delle diverse istanze, è stato utilizzato un cluster messo a disposizione dal Dipartimeto di Ingegneria dell'Informazione 
dell'università. Nello specifico, la macchina in questione presenta le seguenti specifiche tecniche:
\begin{itemize}
\item Intel(R) Xeon(R) CPU E5-2623 v3 @ 3.00GHz quad-core
\item 16GB RAM
\item Linux Fedora 37
\item Python 3.10.6
\item IBM ILOG CPLEX Optimization Studio versione 22.1
\end{itemize}
Per garantire la compatibilità tra l'ultima versione del software CPLEX installata (22.1) e quella di Python, è stato fatto uso di \textit{pyenv} \cite{pyenv}, 
un'utility per Linux e MacOS che ci permette di tenere all'interno dello stesso sistema operativo differenti versioni dell'interprete Python. In questo modo 
è stato possibile utilizzare la versione Python richiesta da CPLEX (3.10) e non quella predefita del cluster (3.11).

Inoltre, sono state installate localmente tutte le librerie che vengono utilizzate all'interno dei vari script, che si ricordano 
essere: \textit{Numpy}, \textit{Pyomo}, \textit{Matplotlib} e \textit{Seaborn}.

\newpage
\section{Tempi medi di risoluzione}
I tempi medi ricavati come illustrato nelle sezioni precedenti possono essere riportati in una tabella


\section{Soluzioni trovate}