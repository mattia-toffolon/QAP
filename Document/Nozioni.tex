In questo primo capitolo sono stati riportati alcuni concetti che rappresentano le fondamenta teoriche del
lavoro svolto. Come prima cosa è stato definito formalmente il concetto di problema di ottimizzazione. Successivamente
si è proseguito con una coincisa trattazione relativa a due importanti algoritmi usati nella risoluzione di 
problemi di ottimizzazione: \textit{branch-and-bound} e \textit{branch-and-cut}. Infine, è stata esposta la
definizione formale di problema di assegnamento quadratico sul quale questo lavoro è stato basato. Molte delle definizioni in questo capitolo sono ispirate alle 
dispense del corso di \textsl{Modelli e Software per l'Ottimizzazione Discreta}, tenuto dal professore 
Domenico Salvagnin \cite{salvagnin1}\cite{salvagnin2}.

\section{Problema di ottimizzazione}
Un problema di ottimizzazione può essere formulato come 
\begin{align}
	\label{eq:opt}
	\begin{array}{l}
      \text{min(\textit{or} max)}f(x)\\
      S	\\
      x \in D
    \end{array}
\end{align}
dove $f(x)$ è una funzione a valori reali nelle variabili $x$, $D$ è il dominio di $x$ e $S$ un insieme finito di vincoli. In generale,  
$x$ è una tupla ($x_1,...,x_n$) e $D$ è un prodotto cartesiano $D_1 \times ... \times D_n$, e vale $x_j \, \in \, D_j$. 

In generale, un problema di ottimizzazione nella forma (\ref{eq:opt}) è intrattabile, nel senso che non esistono
algoritmi efficienti (o addirittura algoritmi) per la sua risoluzione.
Si rende pertanto necessario considerare casi particolari di questa formulazione che presentano struttura e proprietà
particolari da poter sfruttare.

\newpage
\section{Programmazione lineare intera}
Uno dei casi particolari della formulazione generale di problema di ottimizzazione (\ref{eq:opt}) è quello della programmazione lineare intera.
Un problema di programmazione lineare intera consiste nella minimizzazione (o massimizzazione) di una funzione lineare soggetta ad un 
numero finito di vincoli lineari, con in aggiunta il vincolo che alcune delle variabili del problema debbano assumere valori interi. 
In generale, il problema può quindi essere riformulato come:
\begin{align*}
	\begin{array}{l}
      \text{min} \, cx\\
      a_i x \sim b_i \qquad\qquad i=1,...,m \\
      l_j \leq x_j \leq u_j \qquad j=1,...,n =N \\
      x_j \in \mathbb{Z}  \;\,\qquad\qquad \forall j \in J \subseteq N = {1,...,n}	
	\end{array}
\end{align*}
\indent
Se $J=N$ si parla di programmazione lineare intera pura, altrimenti di programmazione lineare intera mista (o MIP, dall'inglese 
\textit{Mixed Integer Programming}).

La programmazione lineare intera restringe quindi notevolmente la tipologia di vincoli a disposizione nel
processo di formalizzazione matematica del problema, determinando una maggior difficoltà in fase di modellazione.
Tuttavia, nel caso della MIP, questa restrizione non comporta un'eccessiva limitazione sul tipo di 
problemi formulabili secondo questo paradigma. Alcuni esempi classici di problemi risolvibili mediante MIP sono \textit{knapsack}, problemi di 
\textit{scheduling}, \textit{minimum vertex cover} e, naturalmente, \textit{quadratic assignment problem}.
Inoltre, l'introduzione dei vincoli di linearità ed interezza comporta notevoli vantaggi nella definizione ed
implementazione di algoritmi di risoluzione, due dei quali sono stati descritti più nel dettaglio nelle seguenti sezioni.

\subsection{Algoritmo Branch and Bound}
L'algoritmo branch-and-bound (B\&B) è un algoritmo di ottimizzazione generica basato sull'enumerazione dell'insieme delle soluzioni
ammissibili di un problema di ottimizzazione combinatoria, introdotto nel 1960 da A. H. Land e A. G. Doig \cite{10.2307/1910129}. 

Questo algoritmo permette di gestire il problema dell'esplosione combinatoria scartando intere porzioni dello spazio delle soluzioni attraverso operazioni di \textit{pruning}.
Ciò risulta possibile quando si riesce a dimostrare che questi sottospazi non possono contenere soluzioni migliori di 
quelle note. Branch-and-bound implementa inoltre una strategia \textit{divide and conquer}, che permette di ottenere la soluzione al problema
ricombinando quelle relative a partizioni del problema stesso. Viene riportata di seguito una breve descrizione dell'algoritmo.

Sia $F$ l'insieme delle soluzioni ammissibili di un problema di minimizzazione (oppure di massimizzazione, a meno di un 
cambio di segno della funzione obiettivo), $c \, : \, F \, \rightarrow \mathbb{R} $ la funzione obiettivo e $\Bar{x} \in F$ una 
soluzione ammissibile nota, generata mediante euristiche o mediante assegnazioni casuali. Il costo di tale soluzione
nota $z=f(\Bar{x})$, detto \textit{incumbent}, rappresenta per sua natura un limite superiore al valore della soluzione ottima. 

L'algoritmo branch-and-bound prevede una fase iniziale di \textit{bounding} in cui uno o più vincoli del problema vengono rilassati, allargando di 
conseguenza l'insieme delle possibili soluzioni $G \supseteq F$. La soluzione di questo rilassamento, se esiste,
rappresenta un \textit{lower bound} alla soluzione ottima del problema iniziale. Se la soluzione di tale rilassamento appartiene a $F$
o ha costo uguale all'attuale \textit{incumbent}, l'algoritmo termina in quanto si è trovata una soluzione ottima del problema.
Se il rilassamento dovesse risultare impossibile, è possibile anche in questo caso terminare la ricerca di una soluzione in quanto si può 
concludere che anche il problema di partenza è impossibile.

Invece, nel caso in cui una soluzione al rilassamento esiste ma non è contenuta nell'insieme delle soluzioni ammissibili $F$,
l'algoritmo procede con l'identificare una separazione $F^*$ di $F$, ossia un insieme finito di sottoinsiemi tale che:
\begin{align*}
\bigcup_{F_i \in F^*} F_{i} = F
\end{align*}
Tale fase, detta di \textit{branching}, è giustificata dal fatto che la soluzione ottima del problema è data dalla minima
tra le soluzioni delle varie separazioni $F_i \in F^*$ dette figli di $F$.
\begin{align*}
    min \blbrace c\( x\) |\; x \in F \brbrace = min\blbrace min\blbrace c\( x\) | \; x \in F_i \brbrace |\; F_i \in F^* \brbrace
\end{align*}
$F^*$ è spesso, anche se non necessariamente, una partizione dell'insieme 
iniziale $F$. A questo punto, tutti i figli di $F$ vengono aggiunti alla coda dei sottoproblemi da processare.

L'algoritmo procede quindi con il selezionare un sottoproblema $P_i$ dalla coda e risolverne un rilassamento. Si possono presentare quattro casi differenti:
\begin{itemize}
\itemsep-0.5em 
\item Se si trova una soluzione $\in F$ migliore dell'attuale incumbent, quest'ultimo viene 
sostituito dalla soluzione trovata e si procede con lo studio di un altro sottoproblema.
\item Se il rilassamento del sottoproblema non ammette soluzione, allora si smette di esplorare l'intero sottoalbero a lui associato 
nello spazio di ricerca (\textit{pruning by infeasibility}).
\item Altrimenti, si confronta la soluzione trovata con il valore corrente dell'\textit{upper-bound} dato dall'incumbent; se
quest'ultimo è minore della soluzione del rilassamento trovata, è possibile anche in questo caso smettere di esplorare il sottoalbero associato al
sottoproblema corrente, in quanto non può portare ad una soluzione migliore di quella già nota (\textit{pruning by optimality}).
\item Infine, se non è stato in alcun modo possibile scartare o concludere l'esplorazione del sottoalbero associato a $P_i$, è necessario
eseguire nuovamente il \textit{branching}, aggiungendo i nuovi sottoproblemi alla lista dei sottoproblemi da processare.
\end{itemize}
L'algoritmo prosegue processando sottoproblemi finché la lista di questi non si svuota. Quando ciò avviene, la soluzione
rappresentata dall'attuale \textit{incumbent} è la soluzione ottima al problema iniziale.

Quella appena descritta rappresenta una formulazione generica dell'algoritmo B\&B. Questa formulazione può essere
tuttavia specializzata nella risoluzione di problemi MIP in maniera immediata, agendo sulle condizioni che regolano 
\textit{bounding} e \textit{branching}. Per quanto riguarda il primo, la scelta più diffusa consiste nel considerare
il rilassamento lineare dei sottoproblemi, rilassando dunque il vincolo di interezza.
Se la soluzione del rilassamento non è intera, una possibile separazione in sottoproblemi può essere effettuata considerando la partizione:
\begin{align*}
x_j \leq \floor*{x_j^*} \vee x_j \geq \ceil*{x_j^*}
\end{align*}
\indent
Infine, è importante notare come, per costruzione, ogni soluzione trovata dall'algoritmo è migliore dell'incumbent e, di conseguenza, 
l'andamento dell'upper bound del problema decresce fino al raggiungimento della soluzione ottima. A differenza degli upper bound, i lower bound 
dei singoli sottoproblemi non hanno invece valenza globale. Nonostante ciò, è comunque possibile derivare un lower bound globale considerando il minimo 
tra tutti i lower bound dei sottoproblemi ancora aperti. Avere a disposizione in qualsiasi momento entrambi i bound del problema, 
permetta quindi di valutare la bontà della soluzione provvisoria in ogni momento. 

\subsection{Algoritmo Branch and Cut}
L'algoritmo \textit{branch-and-cut} (B\&C) rappresenta una versione migliorata dell'algoritmo branch-and-bound, introdotta nel 1987
da M. Padberg e G. Rinaldi \cite{PADBERG19871} e ideata appositamente per la risoluzione di problemi MIP. 

L'algoritmo branch-and-cut è un ibrido tra branch-and-bound, trattato nella sezione precedente, e un algoritmo
a piani di taglio puro, in cui la soluzione è viene ottenuta mediante raffinazioni progressive dello spazio delle soluzioni attraverso la
progressiva aggiunti di vincoli.
Queste due tecniche si rafforzano a vicenda, contribuendo al raggiungimento di prestazioni complessive superiori a quelle
che otterrebbe ciascuna di esse singolarmente.

L'idea alla base di questo algoritmo è quella di "rafforzare" la formulazione associata al rilassamento lineare di ogni sottoproblema 
mediante la generazione di piani di taglio. I vantaggi a livello risolutivo sono molteplici, tra cui una maggior probabilità di
ottenere soluzioni intere del rilassamento lineare o, in alternativa, di ottenere lower bound più stretti e pertanto più efficienti in 
fase di pruning. 

Nonostante l'idea alla base di questo approccio sia relativamente semplice, l'implementazione dell'algoritmo B\&C è tutt'altro che
banale e richiede l'esistenza di procedure efficienti per la risoluzione del seguente problema di \textsl{separazione}: data una
soluzione frazionaria {$x^*$}, trovare una diseguaglianza valida $\alpha^Tx\leq \alpha_0$, se esiste, violata da $x^*$, cioè
tale che $\alpha^Tx^*>\alpha_0$.

\section{Quadratic assignment problem}
I QAP (Quadratic assignment problem), conosciuti anche come \textit{problemi di assegnamento quadratico}, si fondano sulla ricerca
di un posizionamento ottimale di un dato insieme di unità, anche dette \textit{facilities} in riferimento ai \textit{facility location problems}
di cui i QAP sono una branca, in un dato insieme di posizioni, anche dette \textit{locations}. Sono noti i flussi di informazione che
ogni unità deve scambiare con le altre e le distanze tra le varie posizioni. Un posizionamento è ottimo se è minima la somma
dei prodotti tra distanza fra due posizioni e flusso di informazione tra le unità presenti in tali posizioni. 
Come provato da S. Sahni e T. Gonzalez nel 1976 \cite{SAHNI1976}, un problema di assegnamento quadratico è di tipo NP-difficile, 
ciò significa che non esistono algoritmi in grado di trovarne una soluzione in un tempo polinomiale. 
Inoltre, può essere risolto all'ottimo solo per instanze particolarmente piccole.