La ricerca operativa, in inglese \textit{operational research}, è una branca della matematica applicata 
che si occupa dell'analisi e risoluzione di complessi problemi decisionali mediante modelli matematici e 
metodi quantitativi. Lo scopo di questa disciplina è dunque quello di fornire supporto nell'attività 
decisionale in cui occorre gestire e coordinare risorse limitate sottoforma di soluzione al problema ottima, 
quando possibile, o ammissibile. Essa rappresenta un approccio scientifico alla risoluzione di problemi 
complessi che ha trovato largo successo in molti settori come l'economia, l'informatica e l'ingegneria.

Una nota branca della ricerca operativa è l'ottimizzazione. Essa si fonda sulla risoluzione di problemi di 
massimizzazione o minimizzazione di una data funzione, detta funzione obbiettivo, le cui variabili sono legate 
fra loro tramite un dato insieme di vincoli.

Questo studio verte su una particolare classe di problemi di ottimizzazione detta di assegnamento quadratico,
in inglese \textit{quadratic assignment problem} (QAP). In particolare, è stato analizzato l'andamento della
complessità di risoluzione delle istanze del problema al variare della sua dimensione e di altri parametri 
utili alla sua generazione.

La stesura di questo elaborato è stata articolata in quattro capitoli. Inizialmente verranno brevemente esposti
alcuni concetti teorici utili alla comprensione degli esperimenti svolti. Successivamente si proseguirà con le
metodologie adottate per la realizzazione e risoluzione delle istanze e la presentazione dei risultati. Infine,
verrà riportata una breve trattazione relativa alle conclusioni del lavoro svolto.

Tutto il codice sviluppato relativamente a questo studio è stato scritto in Python \cite{python}, linguaggio di 
programmazione \textit{general-purpose} di alto livello che ha permesso di gestire ognuna delle diverse fasi in 
cui si è articolato lo svolgimento degli esperimenti, dalla generazione delle istanze all'elaborazione grafica dei 
risultati. Il codice sorgente, così come tutti i risultati delle elaborazioni, sono liberamente consultabili 
online nella \textit{repository} utilizzata per gestire il controllo di versione del progetto \cite{repository}.